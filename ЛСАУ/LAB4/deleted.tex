% шестое задание почти выполненное по пирнципу внутренней модели
Выполнять её я буду исходя из принципа внутренней модели, рассмотренного на лекции.\ 

Пусть есть передаточная функция объекта управления $W_s(s) = \frac{N(s)}{D(s)}$, передаточная функция регулятора $ H(s) = \frac{N_{\text{рег}}(s)}{D_{\text{рег}}(s)}$, передаточная функция задающего воздействия $G(s) = \frac{N_g(s)}{D_g(s)}$. Тогда передаточная функция к ошибке:

$$
E = \frac{1}{1 + \frac{N_{\text{рег}}(s)}{D_{\text{рег}}}\cdot\frac{N}{D}}\cdot\frac{N_g}{D_g} = \frac{D_{\text{рег}}D}{D_{\text{рег}}D + N_{\text{рег}}N}\cdot\frac{N_g}{D_g}.
$$\

Принцип внутренней модели гласит, что для успешного слежения необходимо, чтобы полюса разомкнутой системы включали в себя полюса задающего воздействия. Значит, задача синтеза регулятора с успешным слежением сводится к выбору $D_{\text{рег}}$, такого,чтобы корни выражения $D_{\text{рег}}D$ включали в себя корни $D_g$. Далее, для получения устойчивости системы выбираем $N_{\text{рег}}$ так, чтобы корни знаменателя первого множителя в E были отрицательными.\ 

Для моего случая:\

$$D_g(s) = 0.45^2 + s^2 \Rightarrow s_{1, 2} = \pm 0.45i$$

$$
D = s^2+5s+6 \Rightarrow s_1 = -3, s_2 = -2
$$\ 

Значит, $D_{\text{рег}}$ должен иметь корни $s_{1, 2} = \pm 0.45i$. Для достижения таких корней можно взять $a_0 = 0.45^2, a_1 = 0, a_2 = 1$. Теперь подберу $N_{\text{рег}}$.\

$$N(s) = 5.$$\ 

$$D_{\text{рег}}D + N_{\text{рег}}N = 0 \Leftrightarrow (s^2+0.45^2)(s^2+5s+6) + 5N_{\text{рег}} = 0.$$

$$(s^2+0.45^2)(s^2+5s+6) + 5N_{\text{рег}} = s^4 + 5s^3 + (6+0.45^2)s^2 + 0.45^2\cdot5s+0.45^2\cdot 6 + 5N_{\text{рег}} = 0$$\ 

Регулятор, при котором выполняется равенство:

$$N_{\text{рег}} = -\frac{s^4 + 5s^3 + (6+0.45^2)s^2 + 0.45^2\cdot5s+0.45^2\cdot 6}{5}.$$\

Регулятор, при котором корни получаются отрицательными:

$$N_{\text{рег}} = -\frac{s^4 + 5s^3 + (6+0.45^2)s^2 + 0.45^2\cdot5s+0.45^2\cdot 6}{5} + (s+1).$$\

Тогда примем $m = 4, b_4 = -\frac{1}{5}, b_3 = -1, b_2 = -\frac{6+0.45^2}{5}, b_1 = \left(-\frac{0.45^2\cdot5}{5}+1\right), b_0 = \left(-\frac{0.45^2\cdot6}{5}+1\right)$. При таких значениях параметров регулятор физически нереализуем, так как степень полинома числителя больше степени полинома знаменателя. Можем воспользоваться приближёнными к $0$ значениями параметров $a_3, a_4$, чтобы преодолеть это ограничение. Однако делать этого не придётся, потому что по заданию нужно вычислять типовой характеристический полином Баттерворта для коэффициента подобия $\omega_0$.\
