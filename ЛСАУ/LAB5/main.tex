\documentclass[a4paper]{article}
\usepackage{cmap}
\usepackage[utf8]{inputenc}
\usepackage[T2A]{fontenc}
\usepackage[english,russian]{babel} 
\usepackage[left=15mm, top=15mm, right=15mm, bottom=42mm, nohead, nofoot]{geometry}
\usepackage{blindtext}  % рыба-текст
\usepackage{graphicx}  % изобржаения
\usepackage{float} % плавающие объекты
\usepackage{wrapfig}  % изобржаения
\usepackage{tikz} % графика
\usepackage{xcolor} % определение цветов
\usepackage{nicefrac} % красивые дроби
\usepackage{cancel} % сокращение
\usepackage{amsmath,amsfonts,amssymb} % математический пакет
\usepackage{hyperref}  % гиперссылки
\usepackage{fancybox,fancyhdr} % хедер и футер
\usepackage{listings} % код
\usepackage{accsupp}
\usepackage{caption}

\captionsetup[figure]{name=Рисунок}
\pagestyle{fancy}
\fancyhf{}
\fancyhead[L]{Лабораторная работа №5}
\fancyhead[R]{\textit{Типовые динамические звенья}}
\fancyfoot[C]{\thepage}
\headsep=8mm
\footskip=20mm

\definecolor{urlcolor}{HTML}{3454D1}
\definecolor{linkcolor}{HTML}{3454D1}
\hypersetup{pdfstartview=FitH, linkcolor=linkcolor, urlcolor=urlcolor, colorlinks=true}

\definecolor{strings}{rgb}{0,0.6,0}
\definecolor{comments}{rgb}{0,0.3,0}
\definecolor{numbers}{rgb}{0.5,0.5,0.5}
\definecolor{keywords}{rgb}{0.09,0.61,0.95}
\definecolor{background}{rgb}{0.97,0.97,0.97}
\newcommand{\noncopynumber}[1]{%
    \BeginAccSupp{method=escape,ActualText={}}%
    #1%
    \EndAccSupp{}%
}
\lstdefinestyle{codestyle}{
    backgroundcolor=\color{background},
    commentstyle=\color{comments},
    keywordstyle=\color{keywords},
    stringstyle=\color{strings},
    numberstyle=\tiny\color{numbers}\noncopynumber,
    basicstyle=\ttfamily\footnotesize,
    breakatwhitespace=false,
    breaklines=true,
    captionpos=b,
    inputencoding=utf8,
    keepspaces=true,
    numbers=left,
    numbersep=5pt,
    showspaces=false,
    showstringspaces=false,
    showtabs=false,
    tabsize=2,
    extendedchars=true,
    literate=
    {а}{{\cyra}}1
    {б}{{\cyrb}}1
    {в}{{\cyrv}}1
    {г}{{\cyrg}}1
    {д}{{\cyrd}}1
    {е}{{\cyre}}1
    {ж}{{\cyrzh}}1
    {з}{{\cyrz}}1
    {и}{{\cyri}}1
    {й}{{\cyrishrt}}1
    {к}{{\cyrk}}1
    {л}{{\cyrl}}1
    {м}{{\cyrm}}1
    {н}{{\cyrn}}1
    {о}{{\cyro}}1
    {п}{{\cyrp}}1
    {р}{{\cyrr}}1
    {с}{{\cyrs}}1
    {т}{{\cyrt}}1
    {у}{{\cyru}}1
    {ф}{{\cyrf}}1
    {х}{{\cyrh}}1
    {ц}{{\cyrc}}1
    {ч}{{\cyrch}}1
    {ш}{{\cyrsh}}1
    {щ}{{\cyrshch}}1
    {ъ}{{\cyrhrdsn}}1
    {ы}{{\cyrery}}1
    {ь}{{\cyrsftsn}}1
    {э}{{\cyrerev}}1
    {ю}{{\cyryu}}1
    {я}{{\cyrya}}1
    {А}{{\CYRA}}1
    {Б}{{\CYRB}}1
    {В}{{\CYRV}}1
    {Г}{{\CYRG}}1
    {Д}{{\CYR96}}1
    {Е}{{\CYRE}}1
    {Ж}{{\CYRZH}}1
    {З}{{\CYRZ}}1
    {И}{{\CYRI}}1
    {Й}{{\CYRISHRT}}1
    {К}{{\CYRK}}1
    {Л}{{\CYRL}}1
    {М}{{\CYRM}}1
    {Н}{{\CYRN}}1
    {О}{{\CYRO}}1
    {П}{{\CYRP}}1
    {Р}{{\CYRR}}1
    {С}{{\CYRS}}1
    {Т}{{\CYRT}}1
    {У}{{\CYRU}}1
    {Ф}{{\CYRF}}1
    {Х}{{\CYRH}}1
    {Ц}{{\CYRC}}1
    {Ч}{{\CYRCH}}1
    {Ш}{{\CYRSH}}1
    {Щ}{{\CYRSHCH}}1
    {Ъ}{{\CYRHRDSN}}1
    {Ы}{{\CYRERY}}1
    {Ь}{{\CYRSFTSN}}1
    {Э}{{\CYREREV}}1
    {Ю}{{\CYRYU}}1
    {Я}{{\CYRYA}}1
}

\lstset{style=codestyle}

\addto\captionsrussian{
  \renewcommand{\contentsname}
    {\centering Содержание}
}


\newlength{\tempheight}
% \newcommand{\Let}{
% \mathbin{\text{\settoheight{\tempheight}{\mathstrut}\raisebox{0.4\pgflinewidth}{
% \tikz[baseline=0.5ex,line cap=round,line join=round] \draw (0,0) --++ (0.3em,0) --++ (0,2.3ex) --++ (-0.3em,0);
% }}}}
\newcommand*\squared[1]{\tikz[baseline=(char.base)]{
            \node[shape=rectangle,draw,inner sep=4pt] (char) {#1};}}
\newcommand*\msquared[1]{\tikz[baseline=(char.base)]{
            \node[shape=rectangle,draw,inner sep=4pt] (char) {$\displaystyle #1$};}}
\newcommand{\at}{\biggr\rvert}
\newcommand{\shiftright}[3]{\makebox[#2][r]{\makebox[#1][l]{#3}}}
\newcommand{\e}{\;\text{e}}
\let\oldint\int
\def\int{\oldint\limits}
\DeclareRobustCommand{\divby}{%
  \mathrel{\vbox{\baselineskip.65ex\lineskiplimit0pt\hbox{.}\hbox{.}\hbox{.}}}%
}

\newcommand\NB{\textbf{N\kern-0.32em\textcolor{red}{B}}}

\begin{document}

\begin{titlepage}
    \begin{center}
        Федеральное государственное автономное образовательное \\ учреждение высшего образования \\[6pt]
        САНКТ-ПЕТЕРБУРГСКИЙ НАЦИОНАЛЬНЫЙ \\ ИССЛЕДОВАТЕЛЬСКИЙ УНИВЕРСИТЕТ ИТМО \\[16pt]
        Факультет систем управления и робототехники \\[26em]
        Лабораторная работа №5\\[0.5em]
        \textbf{ТИПОВЫЕ ДИНАМИЧЕСКИЕ ЗВЕНЬЯ}
    \end{center}\,\\[10em]
    \begin{flushright}
        Студент: Заводин Е.Ю.\\
        Лин САУ R23 бак 1.1.1 \\[0.5em]
        Преподаватели: Перегудин А.А.\\
        Пашенко А.В.
    \end{flushright}\,\\[6em]
    \begin{center}
        {\small Санкт-Петербург \\ 2025}
    \end{center}
\end{titlepage}
\setcounter{page}{2}
\tableofcontents\newpage

\section{Задача исследования типовых динамических звеньев}\

В работе исследуются реальные объекты --- находятся их передаточные функции, сопоставляются с типовыми звеньями, временные и частотные характеристики объектов моделируются и сравниваются с конкретными теоретическими для найденных типовых звеньев. 

\subsection{ДПТ}\

Рассмотрим ДПТ независимого возбуждения, задаваемый формулами

\[
J\dot{\omega} = M, M = k_mI, I = \frac{U + \varepsilon_i}{R}, \varepsilon_i = -k_e\omega. 
\]\

Считая $U$ входом, $\omega$ --- выходом, сведу формулы к одному линейному дифференциальному уравнению:

\[
J\dot{\omega} = k_m \frac{U-k_e\omega}{R} = \frac{Uk_e}{R} - \frac{k_mk_e\omega}{R}
\]

\[
J\dot{\omega} + \frac{k_mk_e \omega}{R} = \frac{k_m}{R} U
\]

\[
\frac{JR}{k_m}\dot{\omega} + k_e\omega = U
\]

\[
\frac{\omega}{U} = W(s) = \frac{1}{\frac{JR}{k_m}s + k_e} = \frac{1}{k_e} \frac{1}{\frac{JR}{k_ek_m}s + 1}
\]

\[
W(s) = \frac{k}{Ts+1}, T = \frac{JR}{k_ek_m}, K = \frac{1}{k_e}
\]\ 

Получил передаточную функцию в стандартизированном виде, соответствующую реальному усилительному звену. Выделю действительную и мнимую части этой передаточной функции:

\[
W(s) = \frac{k}{Ts+1} \Rightarrow W(j\omega) = \frac{k}{T(j\omega)+1}
\]

\[
W(j\omega) = \frac{k(T(j\omega)-1)}{(T(j\omega)+1)(T(j\omega)-1)} = \frac{-K}{-1-\omega^2} + i\frac{KT\omega}{-1-\omega^2} = \frac{K}{1+\omega^2} + i\frac{-KT\omega}{1+\omega^2}.
\]\

Пусть $P(\omega) = \frac{K}{1+\omega^2}$, $Q(\omega) = \frac{-KT\omega}{1+\omega^2}$. Рассчитаю АЧХ такой передаточной функции:

\[
A(\omega) = \sqrt{\frac{(KT\omega)^2 + K^2}{(1+\omega^2)^2}}.
\]\

Фазово-частотная характеристика может быть найдена по следующей формуле:

\[
\varphi(\omega) = \text{atan}2(Q(\omega), P(\omega)).
\]\ 

У двигателя постоянного тока активное сопротивление обмоток ротора $R$, момент инерции ротора $J$ и конструктивные постоянные $k_e, k_m$ при реальном моделировании являются положительными, также буду считать, что $\omega$ принимает только неотрицательные значения (в отрицательной области для вещественной функции результат преобразования Лапласа будет симметричен результату в положительной области). Исходя из этого, коэффициент усиления и постоянная времени

\[
T = \frac{JR}{k_ek_m} > 0, K = \frac{1}{k_e} > 0.
\]\ 

Тогда числитель действительной части передаточной функции всегда положителен, как и знаменатель, а значит, $P(\omega) > 0$. Числитель мнимой части передаточной функции же всегда отрицателен, а знаменатель всегда положителен, следовательно, $Q(\omega) < 0$. Исходя из этих соображений, комплексное число $W(j\omega) = P(\omega) +jQ(\omega)$ находится в четвёртом квадранте, а значит, вместо $\text{atan}2{(Q(\omega), P(\omega))}$ можем использовать $\arctan{\left(\frac{Q(\omega)}{P(\omega)}\right)}$:

\[
\arctan{\left(\frac{Q(\omega)}{P(\omega)}\right)} = \arctan{\left(\frac{\frac{-KT\omega}{1+\omega^2}}{\frac{K}{1+\omega^2}}\right)} = 
\arctan{\left(\frac{-KT\omega}{K}\right)} = 
\arctan{-T\omega} = -\arctan{T\omega}
\]

\subsection{ДПТ 2.0}\

Рассмотрим уравнения для полной модели ДПТ независимого возбуждения: 

\[
J\dot{\omega} = M, M = k_mI, I = \frac{U + \varepsilon}{R}, \varepsilon = \varepsilon_i + \varepsilon_s, \varepsilon_i = -k_e\omega, \varepsilon_s = -L\dot{I}. 
\]\

Считая $U$ входом, $\omega$ --- выходом, сведём формулы к одному линейному дифференциальному уравнению:

\[
J\dot{\omega} = k_m \frac{U-k_e\omega -L\dot{I}}{R} = \left.\frac{k_m}{R}U - \frac{k_mk_e\omega}{R}\omega - \frac{L\dot{I}k_m}{R}\right|\cdot \frac{R}{k_m}
\]

\[
\frac{JR}{k_m}\dot\omega+k_e\omega + L\dot I = U
\]

\[
J\dot\omega = k_mI \Rightarrow I = \frac{J\dot \omega}{k_m}, \dot I = \frac{J\ddot \omega}{k_m}
\]

\[
\frac{JL}{k_m}\ddot\omega +\frac{JR}{k_m}\dot\omega + k_e\omega = U \Leftrightarrow \ddot \omega +\frac{R}{L}\dot \omega +\frac{k_ek_m}{J}\omega = \frac{k_m}{J}U
\]\

Переведу получившееся уравнение в пространство изображений Лаплпаса:

\[
s^2\Omega(s) +\frac{R}{L}s\Omega(s) + \frac{k_e k_m}{J}\Omega(s) = \frac{k_m}{J}U(s)
\]

\[
W(s) = \frac{U(s)}{\Omega(s)} = \frac{\frac{k_m}{J}}{s^2 + \frac{R}{L}s + \frac{k_ek_m}{J}} = \frac{\frac{1}{k_e}}{\frac{J}{k_ek_m}s^2+\frac{RJ}{k_ek_mL}s + 1}
\]\

\[
W(s) = \frac{K}{T^2s^2 + 2T\xi s + 1}, K = \frac{1}{k_e}, T = \sqrt{\frac{J}{k_ek_m}}, \xi = \frac{TR}{2L}
\]\

Получена передаточная функция в стандартизированном виде, соответствующая колебательному звену. Для выделения действительной и мнимой части сперва перейду к частотной передаточной функции:

\[
W(s) = \frac{K}{T^2s^2 + 2T\xi s + 1} \Rightarrow W(j\omega) = \frac{K}{T^2(j\omega)^2 + 2T\xi (j\omega) + 1}
\]

\[
W(j\omega) = \frac{K(T^2(j\omega)^2 + 1 - 2T\xi (j\omega))}{(T^2(j\omega)^2 + 2T\xi (j\omega) + 1)(T^2(j\omega)^2 + 1 - 2T\xi (j\omega))} = \frac{K(-T^2\omega^2 + 1 - 2T\xi (j\omega))}{(-T^2\omega^2 + 1 + j2T\xi \omega)(-T^2\omega^2 + 1 - j2T\xi \omega)} = 
\]

\[
= \frac{K(-T^2\omega^2 + 1 - 2T\xi (j\omega))}{(1-T^2\omega^2)^2+(2T\xi\omega)^2} = \frac{K-KT^2\omega^2}{(1-T^2\omega^2)^2+(2T\xi\omega)^2} +j\frac{-2KT\xi \omega}{(1-T^2\omega^2)^2+(2T\xi\omega)^2}.
\]\

Пусть $P(\omega) = \frac{K-KT^2\omega^2}{(1-T^2\omega^2)^2+(2T\xi\omega)^2}, Q(\omega) = \frac{-2KT\xi \omega}{(1-T^2\omega^2)^2+(2T\xi\omega)^2}.$ Рассчитаю амплитудно-частотную характеристику:

\[
A(\omega) = \sqrt{\left(\frac{-2KT\xi \omega}{(1-T^2\omega^2)^2+(2T\xi\omega)^2}\right)^2 + \left(\frac{K-KT^2\omega^2}{(1-T^2\omega^2)^2+(2T\xi\omega)^2}\right)^2} = \sqrt{\frac{4K^2T^2\xi^2\omega^2 + K^2-2K^2T^2\omega^2 + K^2T^4\omega^4}{\left((1-T^2\omega^2)^2+(2T\xi\omega)^2\right)^2}}=
\]

\[=
K\sqrt{\frac{4T^2\xi^2\omega^2-2T^2\omega^2 + T^4\omega^4 + 1}{\left((1-T^2\omega^2)^2+(2T\xi\omega)^2\right)^2}} = K\sqrt{\frac{T^2\omega^2(4\xi^2-2 + T^2\omega^2) + 1}{\left((1-T^2\omega^2)^2+(2T\xi\omega)^2\right)^2}}.
\]\

Фазо-частотная характеристика будет определяться следующим образом:

\[
\varphi(\omega) = \text{atan2}(Q(\omega), P(\omega)) = \text{atan2}\left(\frac{-2KT\xi \omega}{(1-T^2\omega^2)^2+(2T\xi\omega)^2}, \frac{K-KT^2\omega^2}{(1-T^2\omega^2)^2+(2T\xi\omega)^2}\right)
\]\

Расположение $W(j\omega) = P(\omega) +jQ(\omega)$ на комплексной плоскости, а значит, и оценка фазовой частотной характеристики определяется в зависимости от того, положительные ли значения принимают $P(\omega)$ и $Q(\omega)$.\

Знаменатели $P(\omega)$ и $Q(\omega)$ всегда положительны. Тогда знаки вещественной и мнимой части $W(j\omega)$ определяются их числителями --- $Q(\omega) < 0$ (т.к. $K, T, \xi > 0$), знак $P(\omega)$ меняется в зависимости от значения $\omega$.\

Рассмотрим $P(\omega) > 0$:

\[
\frac{1-T^2\omega^2}{\left((1-T^2\omega^2)^2+(2T\xi\omega)^2\right)^2} > 0 \Leftrightarrow 1-T^2\omega^2 > 0 \Leftrightarrow \omega< \frac{1}{T}
\]
Тогда $P(\omega) > 0$ при $\omega >\frac{1}{T}$, $P(\omega) = 0$ при $\omega = \frac{1}{T}$. следовательно:\

При $0 \leq \omega < \frac{1}{T}$ $W(j\omega)$ находится в четвертом квадранте, значит, 
\[
\varphi(\omega) = \text{atan2}(Q(\omega), P(\omega)) = \arctan\left(\frac{Q(\omega)}{P(\omega)}\right) = -\arctan\!\left( \frac{2T\xi\omega}{1 - T^2\omega^2} \right).
\]\

При $\omega = \frac{1}{T}$ $W(j\omega)$ $P(\omega) = Q(\omega) = 0 \Rightarrow \varphi = -\frac{\pi}{2}.$\

При $\omega > \frac{1}{T}$ $W(j\omega)$ в третьем квадранте, а значит, 

\[
\varphi(\omega) = \text{atan2}(Q(\omega), P(\omega)) = \arctan\left(\frac{Q(\omega)}{P(\omega)}\right) + \pi = -\arctan\!\left( \frac{2T\xi\omega}{1 - T^2\omega^2} \right) + \pi.
\]
Итого:
\[
\varphi(\omega) = \begin{cases}
    -\arctan\!\left( \frac{2T\xi\omega}{1 - T^2\omega^2} \right), &\omega \in [0, \frac{1}{T}) \\
    -\frac{\pi}{2}, &\omega = \frac{1}{T} \\
    -\arctan\!\left( \frac{2T\xi\omega}{1 - T^2\omega^2} \right) + \pi, &\omega \in (\frac{1}{T}, +\infty) \\
\end{cases}
\]

\subsection{Конденсируй-умножай}\

В задании рассматривается уравнение конденсатора 

\[
I = C \frac{dU}{dt}
\]
с $I(t)$ в качестве входа и $U(t)$ в качестве выхода.

Для вывода передаточной функции представлю уравнение в виде

\[
C\dot U = I
\]

Тогда передаточная функция выглядит следующим образом:

\[
W(s) = \frac{1}{Cs}
\]

Она представима в стандартизированной форме:

\[
W(s) = \frac{K}{s}, K = \frac{1}{C}
\]

Функция сопоставима с идеальным интегрирующим звеном. Ёмкость конденсатора --- величина неотрицательная, соответственно и $K > 0$. Для определения АЧХ и ФЧХ перейду к частотной передаточной функции и разобью её на вещественную и мнимую составляющие:

\[
W(s) = \frac{K}{s} \Leftrightarrow W(j\omega) = \frac{K}{j\omega}
\]

\[
W(j\omega) = \frac{-jK\omega}{(j\omega)(-j\omega)} = \frac{-jK\omega}{\omega^2} = \frac{-jK}{\omega}.
\]

В этом случае действительная составляющая $P(\omega)$ равна $0$, значит, $W(j\omega) = P(\omega) + jQ(\omega)$ --- чисто мнимое число, при этом $Q(\omega) < 0$. Найду АЧХ:

\[
A(\omega) = \sqrt{W(j\omega)^2} = \sqrt{\left(\frac{-jK}{\omega}\right)^2} = \frac{K}{\omega}.
\]

ФЧХ же в этом случае будет определяться так:

\[
\varphi(\omega) = \text{atan2}\left(\frac{-K}{\omega}, 0\right) = -\frac{\pi}{2}.
\]

\subsection{Пружинка}\

В задании рассматривается пружинный маятник, представленный на рисунке:

\begin{figure}[H]
    \centering
    \includegraphics[width=0.65\linewidth]{ex4/scheme.png}
    \caption{Пружинный маятник}
\end{figure}\

Его движение задаётся следующими уравнениями:

\[
F_\text{упр} = -kx, F = ma.
\]

Входом этой системы считается некая внешняя сила $F_\text{ext}$, направленная соосно движению маятника, а выходом --- траектория движения $x(t)$. Так как $a = \ddot x$:

\[
F_{ext}(t) = m \ddot{x} + k x
\]

Переходим к пространству изображений Лапласа:

\[
m s^2 X(s) + k X(s) = F_{\text{ext}}(s)
\]

\[
X(s) \left(m s^2 + k\right) = F_{\text{ext}}(s)
\]

\[
W(s) = \frac{X(s)}{F_{\text{ext}}(s)} = \frac{1}{m s^2 + k} = \frac{1}{k}\frac{1}{\frac{m}{k}s^2 + 1}
\]

В стандартизированной форме:

\[
W(s) = \frac{K}{T^2s^2+1}, K = \frac{1}{k}, T^2 = \frac{m}{k}
\]

Получаем консервативное звено. Найдём АЧХ и ФЧХ, перейдя к частотной передаточной функции:

\[
W(s) = \frac{K}{T^2s^2+1} \Rightarrow W(j\omega) = \frac{K}{T^2(j\omega)^2+1} = \frac{K}{-T^2\omega^2+1}
\]

В этом случае в качестве $W(j\omega)$ получается вещественное число. Тогда АЧХ --- просто его модуль:

\[
A(\omega) = \left|W(j\omega)\right| = \left|\frac{K}{-T^2\omega^2+1}\right|.
\]

Заметно, что АЧХ имеет разрыв в точке $\omega_0 = \frac{1}{T}$, и ФЧХ также не определена на этой частоте из-за наличия резонанса (знаменатель становится равен нулю). При этом частотная передаточная функция положительна, если $0 < \omega \leq \omega_0$, и отрицательна при $\omega > \omega_0$. Имея $Q(\omega) = 0, P(\omega) = \frac{K}{-T^2\omega^2+1}$, получаем:

\[
\text{atan2}(Q(\omega), P(\omega)) = \arctan{\left(\frac{0}{P(\omega)}\right)} = 0.
\]

\[
\begin{cases}
    0, & 0 < \omega \leq \frac{1}{T} \\
    \pi, & \omega > \frac{1}{T}.
\end{cases}
\]

\subsection{Что ты такое?}\

В задании рассматривается схема регулятора на операционном усилителе, представленная на рисунке:

\begin{figure}[H]
    \centering
    \includegraphics[width=0.65\linewidth]{ex5/scheme.png}
    \caption{Принципиальная схема регулятора на операционном усилителе}
\end{figure}\

Считая входом системы $U_\text{ВХ}(t)$, а выходом --- $U_\text{ВЫХ}(t)$, можем рассмотреть преобразования, выполняемые над входным сигналом более подробно:\

На входном резисторе сопротивлением $R_1$ при подаче напряжения появляется ток $I(t) = \frac{U_\text{ВХ}(t)}{R_1}$. Далее этот ток попадает на конденсатор с отрицательной обратной связью ёмкостью $C$, и, зная что ток, проходящий через конденсатор такой ёмкости, равен $I(t) = C\frac{du(t)}{dt}$, можем рассчитать напряжение на нём: 

\[
u(t) = u(0) + \frac{1}{C}\int_0^{t} I(x)dx.
\]

Воспользуемся полученным ранее значением $I(t) = \frac{U_\text{ВХ}(t)}{R_1}$ и примем начальное условие $u(0) = 0$, тогда $u(t) = \frac{1}{C}\int_0^{t} \frac{U_\text{ВХ}(x)}{R_1}dx = \frac{1}{R_1C}\int_0^{t} U_\text{ВХ}(x)dx$. В пространстве изображений Лапласа по свойству интегрирования после прохождения конденсатора выходом будет являться $Z(s) = \frac{1}{R_1Cs}$ Далее сигнал идёт на последовательно подключенный резистор $R_2$, и выходом по обратной связи будет являться $W(s) = \frac{1}{R_1Cs} + R_2\cdot I(s) = \frac{1}{R_1Cs} + \frac{R_2}{R_1} = \frac{1 + R_2 Cs}{R_1 Cs}$.

Передаточная функция объекта найдена, можно привести её к стандартизированной форме:

\[
W(s) = \frac{1 + R_2 Cs}{R_1 Cs} = 
\]

\section{Вывод по работе}\ 

Выполнив лабораторную работу, я познакомился с задачами стабилизации и слежения, а также с их решениями, посмотрел на простейшие виды регуляторов, выявил связь между ``подвижностью'' системы и порядком астатизма, научился понимать, какой вид сигнала система может отследить с нулевой установившейся ошибкой, научился аналитически выводить статическую ошибку от внешнего воздействия по передаточным функциям, осознал как синтезировать регуляторы 

\section{Приложение А. Код для выполнения заданий}

\subsection*{Листинг 1. Код для выполнения задания 1}

\begin{lstlisting}[caption={Код для построения графиков для задания 1}, language=matlab]
% clear all;
close all;
% plot(t, y)
\end{lstlisting}
\end{document}
